\documentclass{report}
\usepackage{graphicx}
\usepackage{algorithm}
\usepackage{algpseudocode}
\usepackage{graphicx}
\usepackage{float}  % Ajoute ce package en haut du document

\title{Conception de la gestion des collisions}
\author{}
\date{}

\begin{document}

\maketitle

\section{Conception détaillée}

\subsection{Gestion des collisions}

La classe \texttt{Collision} est un \textbf{thread} dédié à la vérification des collisions entre les différents objets du jeu. Elle s'assure que Mario interagit correctement avec le monde environnant, notamment avec les plateformes, les obstacles et les ennemis.

Cette classe \textbf{n'interagit qu'avec les valeurs du modèle}, conformément aux directives du professeur, et ne manipule donc pas directement la vue.

\subsection{Principe de fonctionnement}

Le thread fonctionne en boucle et vérifie à intervalles réguliers (toutes les \textbf{16 ms}) les collisions possibles en fonction de la position actuelle de Mario. Les interactions principales sont :

\begin{itemize}
    \item \textbf{Collision avec le sol ou les plateformes} : vérification de la présence d'un élément solide sous Mario pour permettre ou non sa chute.
    \item \textbf{Collision avec un plafond} : détection d'un obstacle empêchant Mario de continuer son saut.
    \item \textbf{Collision latérale avec des obstacles} : empêche Mario d'avancer s'il rencontre un mur.
    \item \textbf{Collision avec les ennemis} : déclenchement d'actions spécifiques (élimination de l'ennemi si Mario saute dessus, mort de Mario s'il entre en contact latéral avec l'ennemi).
\end{itemize}

\subsection{Calcul des points de collision}

Pour détecter les collisions, on identifie les \textbf{points clés du rectangle de collision} de Mario :

\begin{itemize}
    \item \textbf{Gauche} : \texttt{mario.getPosition().x + mario.getSolidArea().x}
    \item \textbf{Droite} : \texttt{mario.getPosition().x + mario.getSolidArea().x + mario.getSolidArea().width}
    \item \textbf{Haut} : \texttt{mario.getPosition().y - mario.getSolidArea().y * 2}
    \item \textbf{Bas} : \texttt{mario.getPosition().y + mario.getSolidArea().y + mario.getSolidArea().height * 2}
\end{itemize}

Ensuite, ces positions sont transformées en \textbf{lignes et colonnes dans la matrice du jeu}. En divisant les coordonnées par la taille d'une cellule du jeu, on obtient la position de Mario dans la matrice :

\begin{itemize}
    \item \textbf{Ligne du haut (tête de Mario)} : \texttt{ligneTopdanslaMatrice = posTopenY / CONSTANTS.TAILLE\_CELLULE}
    \item \textbf{Ligne du bas (pieds de Mario)} : \texttt{ligneBottomdanslaMatrice = posBottomenY / CONSTANTS.TAILLE\_CELLULE}
    \item \textbf{Colonne gauche du corps} : \texttt{colonneLeftdanslaMatrice = posLeftenX / CONSTANTS.TAILLE\_CELLULE}
    \item \textbf{Colonne droite du corps} : \texttt{colonneRightdanslaMatrice = posRightenX / CONSTANTS.TAILLE\_CELLULE}
\end{itemize}

Après avoir déterminé ces coordonnées, on vérifie deux points essentiels à chaque itération :

\begin{itemize}
    \item \textbf{Point bas gauche} (pied gauche de Mario) : 
    \resizebox{\textwidth}{!}{\texttt{gp.tm.tilesMatrice[ligneBottomdanslaMatrice][colonneLeftdanslaMatrice]}}
    \item \textbf{Point bas droite} (pied droit de Mario) : 
    \resizebox{\textwidth}{!}{\texttt{gp.tm.tilesMatrice[ligneBottomdanslaMatrice][colonneRightdanslaMatrice]}}
\end{itemize}



Ces points permettent de savoir si Mario est en contact avec une plateforme ou s'il doit tomber.

\subsection{Interaction avec la classe Descente}

La classe \texttt{Descente} est responsable de l'application de la gravité à Mario. Elle fonctionne en parallèle avec la classe \texttt{Collision} pour s'assurer que Mario tombe de manière réaliste lorsque rien ne le soutient.

Principales interactions :
\begin{itemize}
    \item \textbf{Descente contrôlée} : lorsque Mario n'a plus de plateforme sous lui, \texttt{Collision} autorise \texttt{Descente} à faire chuter Mario, donc on met l’attribut \texttt{allowedToFallDown} à VRAI.
    \item \textbf{Blocage de la chute} : si \texttt{Collision} détecte une plateforme sous Mario, il s'arrête et \texttt{Descente} est bloqué.
\end{itemize}

\subsection{Algorithme de la gestion de la descente}$$

\begin{algorithm}
\caption{Gestion de la chute de Mario}
\begin{algorithmic}[1]
\State \textbf{Variables:}
\State mario\_position\_y : Position verticale de Mario
\State leSol : Position Y du sol ou d'une plateforme
\State allowedToFallDown : Booléen indiquant si Mario peut tomber
\State GRAVITY : Constante de la force gravitationnelle

\While{le jeu est en cours}
    \State Attendre 16ms
    \If{allowedToFallDown}
        \If{mario\_position\_y < leSol}
            \State mario.setDirection("down")
            \State $mario\_position\_y \gets mario\_position\_y + GRAVITY$
        \Else
            \If{mario\_position\_y \geq leSol$}
                    \State $mario\_position\_y \gets leSol$ \Comment{Mario est replacé sur le sol}
            \EndIf
        \EndIf
    \EndIf
\EndWhile
\end{algorithmic}
\end{algorithm}

\section{Liste des constantes importantes}

\begin{itemize}
    \item \textbf{INT TAILLE\_CELLULE} = 32 (Taille des cellules de la matrice du jeu)
    \item \textbf{INT GRAVITY} = 1 (Force de la gravité appliquée)
    \item \textbf{INT DELAY} = 16 (Intervalle entre chaque itération de vérification)
\end{itemize}

\section{Diagramme UML associé}

\begin{figure}[H]  % Utilisation de [H] pour forcer l'affichage ici
    \centering
    \includegraphics[width=0.8\textwidth]{diagramme.jpg} % Correction du chemin du fichier et utilisation du bon séparateur de chemin
    \caption{Diagramme UML des interactions entre Collision, Mario, Descente et Jumping.}
    \label{fig:uml}
\end{figure}


\section{Problèmes découverts}

\begin{itemize}
    \item \textbf{Problème de téléportation de Mario lors de l'atterrissage} : Une correction a été apportée pour éviter qu'il ne se retrouve à un niveau supérieur lors de la détection des collisions avec le sol. Ce problème a été discuté avec le professeur lors de la dernière séance. En effet, ce n’était qu’un problème d’indice \texttt{j} et \texttt{j + 1}.
    \item \textbf{Gestion des collisions lors du saut} : Si Mario rentre en collision avec une tuile du jeu avec sa tête en sautant, on doit détecter ça (Moi, Hacene, je devrais faire ça dans la semaine).
\end{itemize}

\end{document}

